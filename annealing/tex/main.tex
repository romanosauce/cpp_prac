\documentclass[a4paper,12pt]{article}

\usepackage{cmap}
\usepackage[T2A]{fontenc}
\usepackage[english,russian]{babel}

\usepackage{amsfonts}
\usepackage{amsmath}
\usepackage{amssymb}
\usepackage{dsfont}

\usepackage[left=2cm, right=2cm, top=2cm, 
        bottom=2cm, bindingoffset=0cm]{geometry}

\author{Данилов Роман \\ 421 гр.}
\title{Алгоритм имитации отжига}
\date{\today}

\begin{document}
\maketitle

\section{Формальная постановка задачи}

Дано $n$ независимых работ из множества $W = \{w_1, w_2, \dots, w_n\}$.
Независимость подразумевает независимость по данным.
Также определено множество $M = \{m_1, m_2, \dots, m_k\}$ из $k$ процессров, по которым
необходимо распределить работы.
И определена функция $time$, отображающая множество работ во множество
натуральных чисел и отражающая время выполнения конкретной работы.
\[time: W \rightarrow \mathbb{N}\]

В реализуемой программе входные данные, задающие функцию $time$ и количество процессоров,
представляют собой число $k$ - число процессоров для построения расписания - на первой строке
и последовательность $N$ натуральных чисел, разделенных пробелом,
на второй:
\begin{gather*}
    Input \\ 
    k \in \mathbb{N}, \, k < 100000 \\
    c_i, \, c_i \in \mathbb{N}, \, c_i < 1000, \, i = \overline{1, N}
\end{gather*}

\noindent, где $c_i$ задаёт вычислительную сложность $i$-ой работы

\vspace{5pt}
Назовём расписанием $T$ следующую двойку функций:
\begin{gather*}
    T = (T_1, \, T_2), \text{где} \\
    T_1 : W \rightarrow M, \\
    T_2 : W \rightarrow \mathbb{N^+}
\end{gather*}
Функция $T_1$ отвечает за <<привязку>> работы к процессору, а $T_2$ - за
распределение работ в пределах одного процессора. $T_2(w_i)$ показывает порядковый номер выполнения
работы $w_i$ на процессоре $T_1(w_i)$.

Определим функцию $start$, вычисляющую время начала работы на процессоре:
\begin{gather*}
    start: W \rightarrow \mathbb{N} \\
    start(w) = \sum_{k=1}^{N} time(w_k) \mathds{1}[T_1(w_k)=T_1(w)] \mathds{1}[T_2(w_k)<T_2(w)]
\end{gather*}

Другими словами, функция $start$ равна сумме времен выполнения всех работ на одном процессоре
с порядковыми номерами, меньшими, чем у рассматриваемой работы.

Работа $w \in W$ выполняется без прерываний,
то есть время её завершения всегда равно $start(w) + time(w)$

Расписание корректно, если для каждой работы задано её распределение на процессор и
выполнение работ на процессорах не пересекается. То есть:
\begin{gather*}
    \forall x \in M \, \exists T_1(x),\\
    \neg \exists x \in M (\exists a \in M (x \neq a \wedge T_1(x) = T_1(a)
    \wedge start(x) \leq start(a) \wedge start(a) < start(x)+time(x)))
\end{gather*}

Небходимо для входных данных построить корректное расписание, минимизирующее суммарное время ожидания.
То есть
\[\min_{correct \, T} \sum_{m \in M} (T_2(m)+time(m))\]

\end{document}
